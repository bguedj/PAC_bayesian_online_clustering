\documentclass[10pt]{beamer}

\usetheme[progressbar=frametitle]{metropolis}
\usepackage{appendixnumberbeamer}
\usepackage{indentfirst}
\usepackage{booktabs}
\usepackage[scale=2]{ccicons}
\usepackage{bbm}
\usepackage{mathrsfs}
\usepackage{dsfont}
\usepackage{pgfplots}
\usepgfplotslibrary{dateplot}

\usepackage{xspace}
\newcommand{\themename}{\textbf{\textsc{PAC-Bayesian Online Clustering}}\xspace}

\title{PAC-Bayesian Online Clustering}
\subtitle{Le li, Benjamin Guedj, Sébastien Loustau}
% \date{\today}
\date{}
\author{Mathis Linger, Selim Dekali}
\institute{ENSAE}
% \titlegraphic{\hfill\includegraphics[height=1.5cm]{logo.pdf}}

\begin{document}

\maketitle

\begin{frame}{Table of contents}
  \setbeamertemplate{section in toc}[sections numbered]
  \tableofcontents[hideallsubsections]
\end{frame}

\section{Notation}

\begin{frame}[fragile]{Notation}
\begin{itemize}
\item $(x_{t})_{1:T}$ : online dataset, where $x_{t} \in \mathbb{R}^{d}$\\
\item $K_{t}$: nb of clusters\\
\item $\hat{c}_{t} = (\hat{c}_{t,1}, \hat{c}_{t,2},...\hat{c}_{t,K_{t}})$ : clusters location, depending on past information $(x_{s})_{1:(t-1)}$ and 
$(\hat{c}_{s})_{1:(t-1)} $\\

\smallbreak
\item When $x_{t}$ is newly revealed, the instantaneous loss:\\
\qquad $\ell(\hat{c}_{t}, x_{t}) = \underset{1 \leq k \leq K_{t}}{min} |\hat{c}_{t,k} - x_{t}|^{2}_{2}$\\
\end{itemize}
\end{frame}

\begin{frame}[fragile]{Notation}
\begin{itemize}
\item $\mathscr{C} = \bigcup^{p}_{k=1}$\\
\item q: discrete probability distribution on the set [1,p]:={1,..p}\\
for any k $\in$ [1,p], let $\pi_{k}$, the probability distribution on $\mathbb{R}^{dk}$\\
\smallbreak
\item For any c $\in \mathscr{C}$, we define $\pi(c)$, as\\
\qquad $\pi(c)= \sum_{k \in [1,p]} q(k) \mathbbm{1}_{\{ c\in\mathbb{R}^{dk}\}} \pi_{k}(c)$
\end{itemize}
\end{frame}

\begin{frame}[fragile]{Notation}
\begin{itemize}
\item $c \in \mathscr{C}$ a partition of $\mathbb{R}^{d}, \pi \in \mathbb{P}(\mathscr{C})$ a quasi prior over this set\\
\item $\lambda >0$ : inverse temperature parameter\\
\smallbreak
\item At each time t, we observe $x_{t}$ and a random partition $\hat{c}_{t+1} \in \mathscr{C}$ is sampled from the quasi-posterior:\\
$d\hat{\rho}_{t+1} \propto \exp{\{ -\lambda S_{t}(c) \}}
d\pi(c)$\\
\smallbreak
\item Cumulative loss:\\
$S_{t}(c) = S_{t-1}(c) + \ell (c, x_{t}) + \frac{\lambda}{2}
\{ \ell(c, x_{t}) - \ell (\hat{c}_{t},  x_{t}) \}^{2}$\\
\end{itemize}
\end{frame}

\section{Sparcity Regret Bounds}

\begin{frame}{Sparcity Regret Bounds}
\begin{block}{\underline{Algo 1: The PAC-Bayesian online Clustering algorithm}}
1: Input parameters: $p>0, \pi \in \mathscr{P}(\mathscr{C}), \lambda >0 \text{ and }S_{0}=0$\\
2: Initialization: Draw $\hat{c}_{1} \sim \pi$\\
3: For $t \in [1, T-1]:$\\
4: \quad Get the data $x_{t}$\\
5: \quad Draw $\hat{c}_{t+1} \sim \hat{\rho}_{t+1}(c)$ where $d\hat{\rho}_{t+1} \propto \exp{\{ -\lambda S_{t}(c) \}}
d\pi(c)$, and \\
\hspace*{2cm} $S_{t}(c) = S_{t-1}(c) + \ell (c, x_{t}) + \frac{\lambda}{2}
\{ \ell(c, x_{t}) - \ell (\hat{c}_{t},  x_{t}) \}^{2}$\\
6: End for
\end{block}
\end{frame}


\begin{frame}{Sparcity Regret Bounds}
\begin{block}{\underline{Theorem 1:}}
For any $(x_{t})_{1:T} \in \mathscr{R}^{dT}$,
any quasi prior $\pi \in \mathscr{P}(\mathscr{C})$, any $\lambda >0$,\\
\smallbreak 
the procedure described in Algo 1 satisfies:\\
\smallbreak
$
\sum_{t=1}^{T} \mathds{E}_{(\hat{\rho}_{1},\hat{\rho}_{2},...\hat{\rho}_{t})}
\ell(\hat{c}_{t}, x_{t}) \leq  \underset{\rho \in \mathscr{P}_{\pi}(\mathscr{C})}{inf}  \left\{ \mathds{E}_{c \sim \rho} [ \sum_{t=1}^{T} \ell (c, x_{t}) ] +
\frac{\mathcal{K}(\rho, \pi)}{\lambda} + 
\frac{\lambda}{2} \mathds{E}_{(\hat{\rho}_{1},\hat{\rho}_{2},...\hat{\rho}_{T})} \mathds{E}_{c \sim \rho} \sum_{t=1}^{T} [\ell (c, x_{t}) - \ell (\hat{c}_{t}, x_{t})]^{2} \right\}
$
\end{block}
\end{frame}

\begin{frame}[fragile]{Sparcity Regret Bounds}
The regret bound could be refine when :\\
\begin{itemize}
\item $q(k) = \frac{\exp{-\eta k}}{\sum_{i=1}^{p}\exp{-\eta i}}$, with $\eta >0$.\\
\item $d\pi_{k}(c, R) = \left(\frac{\Gamma (\frac{d}{2}+1)}{\pi^{\frac{d}{2}}} \right) \frac{1}{(2R)^{dk}} \left\{ \prod_{j=1}^{k} \mathbbm{1}_{\{ \mathbb{B}_{d}(2R) \} }(c_{j}) \right\} dc$
\smallbreak
\begin{block}{\underline{Corollary 1:}}
$\sum_{t=1}^{T} \mathds{E}_{(\hat{\rho}_{1},\hat{\rho}_{2},...\hat{\rho}_{t})}             \ell (\hat{c}_{t}, x_{t}) \leq                                     \underset{k \in [1,p]}{inf} \left\{                                       \underset{c \in \mathscr{C}(k,R)}{inf} \sum_{t=1}^{T}  \ell (c, x_{t})   +\frac{dk}{2\lambda}\log{\frac{8R^{2}\lambda T}{d=2}}+\frac{\eta}{\lambda}k \right\}
+ \left( \frac{\log{p}}{\lambda} +\frac{d}{2\lambda} +\frac{\lambda T C_{1}^{2}}{2} \right)$
\smallbreak
where $C_{1}=(2R+max_{t=1..T}|x_{t}|_{2})^{2} $
\end{block}
\end{itemize}
\end{frame}

\begin{frame}[fragile]{Sparcity Regret Bounds}
The below calibration yields a sublinear remainder term: :\\
\begin{itemize}
\item $\lambda = \frac{d+2}{2\sqrt{T}R^{2}}$\\
\end{itemize}
\smallbreak
\begin{block}{\underline{Corollary 2:}}
$\sum_{t=1}^{T} \mathds{E}_{(\hat{\rho}_{1},\hat{\rho}_{2},...\hat{\rho}_{t})}             \ell (\hat{c}_{t}, x_{t}) \leq                                     \underset{k \in [1,p]}{inf} \left\{                                       \underset{c \in \mathscr{C}(k,R)}{inf} \sum_{t=1}^{T}  \ell (c, x_{t})   +k\frac{dR^{2}}{d+2}\sqrt{T}\log{4\sqrt{T}}+ k \frac{2R^{2}\eta}{d+2}\sqrt{T} \right\}
+ \left( \frac{2R^{2}\log{p}}{d+2} +\frac{dR^{2}}{d+2} +\frac{(d+2) C_{1}^{2}}{4R^{2}} \right) \sqrt{T}$
\smallbreak
Hence, if there exist $k^{*}$, and $c^{*} \in \mathscr{C}(k^{*}, R)$, which achieve the infimum:
$
\sum_{t=1}^{T} \mathds{E}_{(\hat{\rho}_{1},\hat{\rho}_{2},...\hat{\rho}_{t})} \ell (\hat{c}_{t}, x_{t}) - \sum_{t=1}^{T} \ell (c^{*}, x_{t}) \leq J k^{*}\sqrt{T}\log{T}
$
\smallbreak
J: constant depending on d,R, $\log{p}$ and $C_{1}^{2} $\\
Then the regret of the expected cumulative loss is sublinear in T.
\end{block}
\end{frame}



\section{Adaptative Sparcity Regret Bounds}

\begin{frame}{Adaptative Sparcity Regret Bounds}
T is usually unknown, prompting us to choose $\lambda = \lambda_{t}$
\smallbreak
\begin{block}{\underline{Algo 1: The adaptative PAC-Bayesian online Clustering algorithm}}
1: Input parameters: $p>0, \pi \in \mathscr{P}(\mathscr{C}), (\lambda_{t})_{0:T} >0 \text{ and }S_{0}=0$\\
2: Initialization: Draw $\hat{c}_{1} \sim \pi$\\
3: For $t \in [1, T-1]:$\\
4: \quad Get the data $x_{t}$\\
5: \quad Draw $\hat{c}_{t+1} \sim \hat{\rho}_{t+1}(c)$ where $d\hat{\rho}_{t+1} \propto \exp{\{ -\lambda_{t} S_{t}(c) \}}
d\pi(c)$, and \\
\hspace*{2cm} $S_{t}(c) = S_{t-1}(c) + \ell (c, x_{t}) + \frac{\lambda_{t-1}}{2}
\{ \ell(c, x_{t}) - \ell (\hat{c}_{t},  x_{t}) \}^{2}$\\
6: End for
\end{block}
\end{frame}


\begin{frame}{Adaptative Sparcity Regret Bounds}
\begin{block}{\underline{Theorem 2:}}
For any $(x_{t})_{1:T} \in \mathscr{R}^{dT}$,
any quasi prior $\pi \in \mathscr{P}(\mathscr{C})$, \\
if $(\lambda_{t})_{0:T}$ a non-increasing sequence of positive numbers,\\
\smallbreak 
the procedure described in Algo 2 satisfies:\\
\smallbreak
$
\sum_{t=1}^{T} \mathds{E}_{(\hat{\rho}_{1},\hat{\rho}_{2},...\hat{\rho}_{t})}
\ell(\hat{c}_{t}, x_{t}) \leq  \underset{\rho \in \mathscr{P}_{\pi}(\mathscr{C})}{inf}  \left\{ \mathds{E}_{c \sim \rho} [ \sum_{t=1}^{T} \ell (c, x_{t}) ] +
\frac{\mathcal{K}(\rho, \pi)}{\lambda_{T}} + 
\mathds{E}_{(\hat{\rho}_{1},\hat{\rho}_{2},...\hat{\rho}_{T})} \mathds{E}_{c \sim \rho} \left[ \sum_{t=1}^{T} \frac{\lambda_{t-1}}{2} [\ell (c, x_{t}) - \ell (\hat{c}_{t}, x_{t})]^{2} \right] \right\}
$
\end{block}
\end{frame}



\begin{frame}[fragile]{Adaptative Sparcity Regret Bounds}
keeping previous setting for q and $\pi_{k}$, with $\eta \geq 0$ and $R\geq \underset{t=1..T}{max} |x_{t}|_{2}$\\
The below adaptative calibration for any $t\in [1,T] and \lambda_{0}=1$: :\\
\begin{itemize}
\item $\lambda_{t} = \frac{d+2}{2\sqrt{t}R^{2}}$\\
\end{itemize}
\smallbreak
\begin{block}{\underline{Corollary 2:}}
Then the algorithm 2 satisfies:\\
$\sum_{t=1}^{T} \mathds{E}_{(\hat{\rho}_{1},\hat{\rho}_{2},...\hat{\rho}_{t})}             \ell (\hat{c}_{t}, x_{t}) \leq                                     \underset{k \in [1,p]}{inf} \left\{                                       \underset{c \in \mathscr{C}(k,R)}{inf} \sum_{t=1}^{T}  \ell (c, x_{t})   +\frac{dkR^{2}}{d+2}\sqrt{T}\log{4\sqrt{T}}+ k \frac{2R^{2}\eta}{d+2}\sqrt{T} \right\}
+ \left( \frac{2R^{2}\log{p}}{d+2} +\frac{dR^{2}}{d+2} +\frac{(d+2) C_{1}^{2}}{2R^{2}} \right) \sqrt{T}$
\smallbreak
The adaptative Algorithm 2 is supported by a sparcity regret bound with rate $\sqrt{T}\log{T}$.
\end{block}
\end{frame}


\section{The PACO algorithm}


\begin{frame}[fragile]{The PACO algorithm}
\begin{itemize}
\item Since direct sampling from the quasi-posterior $\hat\rho_{t}$ is isually not possible, we will focus on a stochastic approximation, called PACO.

\item Approximate $\hat\rho_{t}$ through MCMC, favoring local move.

\item States of interest of the MC $(k^{(n)}, c^{(n)})_{0 \leq n \leq N}$,\\ where $k^{(n)} \in [1,p]$ and $c^{(n)} \in \mathds{R}^{dk^{(n)}}$ 

\item At each iteration, from $(k^{(n)}, c^{(n)})$ to proposal state $(k', c')$\\
Hence $c^{(n)} \in \mathds{R}^{dk^{(n)}}$, and $c' \in \mathds{R}^{dk'}$ may be of different dimensions $(k' \neq k^{(n)})$\\
We create auxilary vectors $\nu_{1}, \nu_{2}$ to compensate for dimensional difference $(d_{1}, d_{2})$ s.t. $dk^{(n)} + d_{1} = dk' + d_{2}$\\

\end{itemize}
\end{frame}


\begin{frame}[fragile]{The PACO algorithm}
\begin{itemize}
\item Let $\rho_{k'}(., c_{k'}, \tau_{k'})$ denote the multivariate Student distribution on $\mathds{R}^{dk'}$:\\

$
\rho_{k'}(c, c_{k'}, \tau_{k'}) = \prod_{j=1}^{k'} \left\{       C_{\tau_{k'}}^{-1} \left( 1+ \frac{|c_{j}-c_{k',j}|_{2}^{2}}{6\tau_{k'}^{2}}     \right)^{-\frac{3+d}{2}}  \right\}dc
$,\\
where $C_{\tau_{k'}}^{-1}$ is the normalizing constraint\\

\item First a local move from $k^{(n)}$ to $k'$ is proposed by choosing \\
$k' \in [k^{(n)}-1, k^{n}+1]$ with probability $q(k^{(n)},.)$

\item Next, choosing $d_{1} =dk'$, $d_{2}=dk^{(n)}$, we sample $\nu_{1}$ from $\rho_{k'}$

\item Finally, the pair $(\nu_{2}, c')$ is obtained by $(\nu_{2}, c') = g(\nu_{1}, c^{(n)})$, where\\
$g:(x,y)\in \mathds{R}^{dk'}x\mathds{R}^{dk^{(n)}} \rightarrow (y,x) \in \mathds{R}^{dk^{(n)}}x\mathds{R}^{dk'}$


\end{itemize}
\end{frame}



\begin{frame}{The PACO algorithm}
\begin{block}{\underline{Algo 3: PACO}}
1: Initialization: $(\lambda_{t})$\\
2: For $t \in [1, T-1]:$\\
3: Initialization: $(k^{(0)}, c^{(0)}) \in [1,p] x \mathds{R}^{dk^{(0)}} $\\
4: For $n \in [1, N-1]:$\\
5: \quad Sample $k' \in [k^{n}-1, k{n}+1]$ from $q(k^{(n)}, .)=1/3$\\
6: \quad Let $c' \leftarrow$ standard k-means output.\\
7: \quad Let $\tau' =  1/ \sqrt{pt}$.\\
8: \quad Sample $\nu_{1} \sim \rho_{k'}(.,c_{k'}, \tau_{k'})$ .\\
9: \quad Let $(\nu_{2}, c') = g((\nu_{1}, c^{(n)})$.\\
10: \quad Accept the move $(k^{(n)}, c^{(n)}) = (k',c')$ with probability\\
$\alpha \left[  (k^{(n)}, c^{(n)}) , (k',c') \right]$
$= min \left\{ 1, \frac{\hat{\rho}_{t}(c')q(k', k^{(n)})\rho_{k^{(n)}}(c^{(n)}, c_{k^{(n)}}, \tau_{k^{(n)}})}{\hat{\rho}_{t}(c^{(n)})q(k^{(n)},k') \rho'(c', c_{k'}, \tau_{k'})}   \right\}  $\\
11: \quad Else $(k^{n+1}, c^{n+1}) = (k^{n}, c^{n})$\\
12: End for\\
13: Let $\hat{c}_{t} = c^{(N)}$.\\
14: End for
\end{block}
\end{frame}



\section{Numercial studies}





\end{document}
